\documentclass{article}
\usepackage[utf8]{inputenc}
\usepackage{geometry}[margin=1in]
\usepackage{amsmath}

\title{thesis notes}
\author{julia andronowitz }
\date{2022-2023}

\begin{document}

\maketitle

\section{To Do}

\begin{itemize}
    \item add cost and gamma values to figures
    \item add how kernel fits into solution equation
    \item calibration and regression-like problems
    \item looking for outliers
    \item compare R and SAS SVM modules
    \item e1071 documentation?
\end{itemize}

done
\begin{itemize}
    \item find certain situations where SVM is applied
    \item give concrete examples
    \item start out by talking about machine learning in general
    \item analogy w 2-d for equation of a line to generalize to multidimensional
    \item classification v regression
    \item explain how i made graph
    \item terminology in introduction (features, etc.)
    \item replace remaining photos with better quality
    \item rename accordingly
\end{itemize}

\section{Introductory Notes}

\begin{itemize}
    \item optimal margin: gap between points on each side have smallest distance from separating line; this is what the SVM calculates
    \item linearly separable means there is a separating hyperplane where those values above are positive and below negative
    \item N x k data matrix and N x 1 vector Y that classifies the two species
    \item hyperplane given by \[ f(x)=w \cdot x +b = 0 \]
    \item the vector $w$ is the normal vector to the hyperplane (so wx1 - wx2 = 0)
    \item the distance from a point to the hyperplane is \begin{align*}
        D &= \frac{f(p)}{||w||} \\
        &= ||(p-x_1)||\cos{\theta} \\
        &= \frac{w \cdot (p-x_1)}{||w||}\\
        &=\frac{w \cdot p - w \cdot x_1}{||w||} \\
        &= \frac{wp + b}{||w||} \\
        &= \frac{f(p)}{||w||}
    \end{align*}
    % \item linearly separable if $ \max w \cdot x _{x \in A^_}< \min w \cdot x_{x \in A^+}$
    \item $B^-$ is the max value of the minimums (those above the hyperplane)
    \item $B^+$ min value of the positives (below the hyperplane)
    \item supporting hyperplane: every point is on one side and you can't move it any further over otherwise that point will be on the wrong side
    \item geometric margin: the perpendicular distance between the two supporting hyperplanes
\end{itemize}

\textbf{notes on 3094 book}
\begin{itemize}
    \item maximal margin classifier: separates two distinct classes
    \item support vector classifier / soft margin classifier: separates most data points according to a margin
    \item makes use of slack variables, epsilons that allow some points to be on the wrong side; if $\epsilon=0$ it's on the right side of the margin, $\epsilon>0$ wrong side of margin and $\epsilon>1$ wrong side of hyperplane
    \item For C > 0 no more than C observations can be on the wrong side of the hyperplane, because if an observation
%is on the wrong side of the hyperplane then ϵi > 1, and (9.15) requires
that sum of n
%%i=1 ϵi ≤ C. As the budget C increases, we become more tolerant of
violations to the margin, and so the margin will widen. Conversely, as C
decreases, we become less tolerant of violations to the margin and so the
margin narrows.
    \item support vectors: points that lie on the wrong side of the margin; these are the only observations that affect the classifier
\end{itemize}

\begin{itemize}
    \item SKILEARN alphas are the same as his lambdas
    \item 
\end{itemize}

\end{document}
